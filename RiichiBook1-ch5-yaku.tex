%~~~~~~~~~~~~~~~~~~~~~~~~~~~~~~~~~~~~~~~~~~~~~~~~~
% Riichi Book 1, Chapter 5: Pursuing yaku
%~~~~~~~~~~~~~~~~~~~~~~~~~~~~~~~~~~~~~~~~~~~~~~~~~

\chapter{Pursuing yaku} \label{ch:yaku}
\thispagestyle{empty}

\bigskip
As we saw in the previous chapter, we often face a tradeoff between speed (tile efficiency) and hand value. In modern riichi mahjong, the value of pursuing expensive {\jap yaku} is much diminished because of red fives. 
For example, {\jap ryanpeiko} (Twice Pure Double Sequence) is a beautiful three-{\jap han yaku}, but it is extremely difficult to make this {\jap yaku}. We can achieve the same hand value more easily with riichi + {\jap dora} + one red five. We thus tend to think of expensive {\jap yaku} as something that emerges in a hand (almost) by chance, not something we actively pursue. 
Given that we can get high scores also from {\jap ippatsu}, {\jap ura dora}, and {\jap tsumo}, getting the hand ready for riichi is generally more important than pursuing expensive {\jap yaku}. 

\bigskip
That being said, always trying to maximize tile efficiency without regard for {\jap yaku} is not the best strategy, either. We should thus design a five-block configuration with an eye to possible {\jap yaku} we can reasonably get. Moreover, sometimes situations call for an expensive hand. For example, when you are ranked fourth in South-4, and the player who is currently ranked third has 10000 more points than you do, you should aim for {\jap mangan tsumo} or {\jap haneman ron} to improve the placement (more on this in Chapter \ref{ch:grand}), which will require that your hand has some {\jap yaku} other than just riichi and {\jap dora}. 

\bigskip
In this chapter, I will discuss some tips to get the following five set of {\jap yaku}. 
%\bi
%\i[]\ref{sec:san} {\jap sanshoku} (Mixed Triple Sequence)
%\i[]\ref{sec:itt} {\jap ittsu} (Pure Straight)
%\i[]\ref{sec:pinfu} {\jap pinfu} (Pinfu / Peace)
%\i[]\ref{sec:hon} {\jap honitsu} (Half Flush)
%\i[]\ref{sec:toi} {\jap toitoi} (All Triplets) and {\jap chiitoitsu} (Seven Pairs)
%\ei
\bigskip

\begin{tabular}{l l}
\ref{sec:san} {\jap sanshoku} (Mixed Triple Sequence) & \ref{sec:itt} {\jap ittsu} (Pure Straight)\\
\ref{sec:pinfu} {\jap pinfu} (Pinfu) & \ref{sec:hon} {\jap honitsu} (Half Flush)\\
\multicolumn{2}{l}{\ref{sec:toi} {\jap toitoi} (All Triplets) and {\jap chiitoitsu} (Seven Pairs)}
\end{tabular}


\newpage
\section{How to get {\jap sanshoku}} \index{sanshoku@{\jap sanshoku}} \label{sec:san}

{\jap Sanshoku} (Mixed Triple Sequence) is an elusive {\jap yaku}. 
Even when we make our hand ready for {\jap sanshoku}, we may lose it at the very last minute. For example, suppose we manage to get the hand ready for {\jap sanshoku} of 345, and we have a side-wait protorun {\LARGE\wan{4}\wan{5}} as the final wait. We will get {\jap sanshoku} only if we win the hand on {\LARGE\wan{3}}; if we win on {\LARGE\wan{6}}, we will lose {\jap sanshoku}.

\bigskip
On the other hand, it is possible to have a confirmed {\jap sanshoku}, but doing so often entails a significant loss in tile efficiency. For example, if our wait were a closed-wait protorun {\LARGE\wan{3}\wan{5}} instead, {\jap sanshoku} is confirmed; but, a closed wait of {\LARGE\wan{4}} is not very good. As long as we seek to utilize side waits to maximize tile efficiency, {\jap sanshoku} becomes difficult to achieve. I will discuss the following seven methods to capture this elusive {\jap yaku}. 

\bigskip
\begin{tabular}{l l}
\ref{sec:san1}~Floating & \ref{sec:san2}~Switching\\
\ref{sec:san3}~Double closed shape & \ref{sec:san4}~Stretched single\\
\ref{sec:san5} Lining pairs &\ref{sec:san6} Golden\\
\ref{sec:san7} Crashing a meld\\
\end{tabular}

\newpage

\subsection{Floating}\label{sec:san1}

\begin{itembox}[r]{{\jap Sanshoku} 1}
\bp
\wan{3}\wan{5}\wan{6}\tong{3}\tong{4}\tong{5}\suo{1}\suo{1}\suo{3}\suo{4}\suo{7}\suo{8}\suo{9}\bei
\ep
\vspace{-10pt} What would you discard? \vspace{-5pt}
\end{itembox}
\noindent
{\LARGE\wan{3}} in this hand is essentially a redundant floating tile from a pure tile efficiency perspective; we do not need it to accept {\LARGE\wan{4}}. However, if we keep it and discard {\LARGE\bei} instead, we can hope to get {\jap sanshoku} of 345. The best case scenario is to draw {\LARGE\wan{4}} first, after which we discard {\LARGE\wan{6}}. 

\bigskip
That being said, keeping a floating tile this way is risky. What if an opponent calls riichi after we discard {\LARGE\bei}? We will have to discard either {\LARGE\wan{3}} or {\LARGE\wan{6}} against the riichi'ed player when this hand becomes ready. To make things worse, if we draw {\LARGE\suo{2}} first, we will have to discard the potentially dangerous {\LARGE\wan{3}} (instead of {\LARGE\bei}) with no benefit of getting {\jap sanshoku}. 

\bigskip
Moreover, when we draw {\LARGE\suo{5}} first, a difficult question arises. Consider the following hand. 
\bp
\wan{3}\wan{5}\wan{6}\tong{3}\tong{4}\tong{5}\suo{1}\suo{1}\suo{3}\suo{4}\suo{7}\suo{8}\suo{9}~\suo{5}\\
\hfill\footnotesize{Draw~~~~~~~~~~~~~~~}
\ep
Should we discard {\LARGE\wan{6}} and have a closed-wait hand in hopes of getting {\jap sanshoku}, or should we discard {\LARGE\wan{3}} and give up on {\jap sanshoku} in pursuit of tile efficiency?
If a hand has at least one {\jap dora} or {\jap tanyao} (All Simples), we should discard {\LARGE\wan{3}} to choose a side wait {\jap pinfu} hand. Only when there is no other {\jap yaku} or {\jap dora} in a {\jap pinfu} hand, it is OK to choose closed-wait {\jap sanshoku} over side-wait {\jap pinfu}.\footnote{We will talk more about a tradeoff of this kind in Chapter \ref{ch:riichi}.}


%\vfill
%\begin{itembox}[c]{{\jap sanshoku}: floating}
%Keeping a floating tile to aim for {\jap sanshoku} is risky. Give it up and pursue tile efficiency if a hand already has two {\jap han} or more.
%\end{itembox}
%\clearpage

\newpage
\subsection{Switching} \label{sec:san2}
\index{1-away (1-{\jap shanten})!perfect 1-away}

\begin{itembox}[r]{{\jap Sanshoku} 2}
\bp
\wan{2}\wan{3}\wan{4}\wan{6}\wan{6}\tong{7}\tong{8}\tong{8}\suo{2}\suo{3}\suo{4}\suo{4}\suo{5}~\tong{3}\\
\hfill\footnotesize{Draw~~~~~~~~~~~}
\ep
\vspace{-17pt}What would you discard? \vspace{-5pt}
\end{itembox}
\noindent
The hand already has five tile blocks and all the blocks are strong; in fact, this is a perfect 1-away hand.\footnote{For the definition of perfect 1-away, see Section \ref{sec:perfect}.} It is thus OK to discard {\LARGE\tong{3}} we just drew. After all, that is the best discard from a tile efficiency perspective. 

\bigskip
However, if we need an expensive hand, we can keep {\LARGE\tong{3}} and discard {\LARGE\tong{8}} instead. The resulting loss in tile efficiency is not very big, as we would still have a strong 1-away hand with two side-wait protoruns: {\LARGE\tong{7}\tong{8}} and {\LARGE\suo{4}\suo{5}}. In addition, {\LARGE\tong{3}} serves as a floating tile to approach {\jap sanshoku} of 234. 
Keep in mind that we should give up on {\jap sanshoku} and do insta-riichi if we draw any of {\LARGE\tong{6}\tong{9}\suo{3}\suo{6}} first (unless you absolutely need {\jap mangan} or above to improve the placement in South-4). 

\bigskip
On the other hand, if we draw {\LARGE\tong{2}} or {\LARGE\tong{4}} first, the hand will be 1-away from ready for {\jap sanshoku}. For example, with a draw of {\LARGE\tong{4}}, the hand becomes the following. 
\bp
\wan{2}\wan{3}\wan{4}\wan{6}\wan{6}\tong{3}\tong{7}\tong{8}\suo{2}\suo{3}\suo{4}\suo{4}\suo{5}~\tong{4}\\
\hfill\footnotesize{Draw~~~~~~~~~~~~~~~}
\ep
We should discard {\LARGE\tong{7}} and then {\LARGE\tong{8}} to aim for {\jap sanshoku} of 234. We are switching from one protorun {\LARGE\tong{7}\tong{8}} to another protorun {\LARGE\tong{3}\tong{4}} to approach {\jap sanshoku}. 

\bigskip
The key here is that we are keeping the hand 1-away throughout the entire process of switching from a perfect 1-away hand to a {\jap sanshoku} 1-away hand. You should not pursue {\jap sanshoku} if switching requires reverting a 1-away hand to 2-away. 


\bigskip
\subsection{Double closed ({\jap ryankan}) shape}\label{sec:san3}
	\index{double closed shape@double closed ({\jap ryankan}) shape} 
	\index{ryankan@{\jap ryankan} (double closed) shape}

\begin{itembox}[r]{{\jap Sanshoku} 3}
\bp
\wan{2}\wan{2}\wan{5}\wan{6}\tong{2}\tong{3}\tong{5}\tong{6}\tong{7}\suo{2}\suo{3}\suo{5}\suo{7}~\tong{1}\\
\hfill\footnotesize{Draw~~~~~~~~~~~}
\ep
\vspace{-17pt}What would you discard? \vspace{-5pt}
\end{itembox}
\noindent
As we have two blocks in {\jap manzu} (cracks) and another two blocks in {\jap pinzu} (dots), we only need one block in {\jap souzu} (bamboos). Our choice is thus between (a) keeping a side-wait protorun {\LARGE\suo{2}\suo{3}} to maximize tile efficiency and (b) keeping a double closed shape {\LARGE\suo{3}\suo{5}\suo{7}} in hopes of getting {\jap sanshoku} of 567. 

\bigskip
If the hand has at least one {\jap dora} or some {\jap yaku} (such as {\jap tanyao}), we should give up {\jap sanshoku} and discard {\LARGE\suo{5}\suo{7}}. Only if the hand has no other {\jap yaku} or {\jap dora}, it is OK to discard {\LARGE\suo{2}} to aim for {\jap sanshoku}. 

\bigskip
Keep in mind, though, that pursuing {\jap sanshoku} with a hand like this is risky, even compared with the floating method we discussed in \ref{sec:san1}. We will end up with a bad-wait {\jap yaku}-less hand if we draw {\LARGE\wan{4}} first. If we give up on {\jap sanshoku} sooner and choose the side-wait protorun {\LARGE\suo{2}\suo{3}}, we can at least get {\jap pinfu}. 

\newpage
\subsection{Stretched single ({\jap nobetan}) shape}\label{sec:san4}
	\index{stretched single shape@stretched single ({\jap nobetan}) shape} 
	\index{nobetan@{\jap nobetan} (stretched single) shape}
\begin{itembox}[r]{{\jap Sanshoku} 4}
\bp
\wan{4}\wan{5}\wan{6}\wan{7}\wan{8}\tong{3}\tong{3}\tong{6}\tong{7}\suo{3}\suo{4}\suo{5}\suo{6}~\suo{7}\\
\hfill\footnotesize{Draw~~~~~~~~~~~}
\ep
\vspace{-17pt}What would you discard? \vspace{-5pt}
\end{itembox}
\noindent
Notice that this hand has two possibilities of {\jap sanshoku}, 567 or 678, and we do not know at this point which one we can get. An excellent way to aim for {\jap sanshoku} with a hand like this is to discard {\LARGE\wan{4}} to have a stretched single shape {\LARGE\wan{5}\wan{6}\wan{7}\wan{8}} that contains both 567 and 678. 

\bigskip
\noindent If we draw any of {\LARGE\tong{5}\suo{2}\suo{5}} first, we discard {\LARGE\wan{8}} to aim for {\jap sanshoku} of 567. 
\bp
\wan{5}\wan{6}\wan{7}\wan{8}\tong{3}\tong{3}\tong{6}\tong{7}\suo{3}\suo{4}\suo{5}\suo{6}\suo{7}~\suo{2}\\
\hfill\footnotesize{Draw~~~~~~~~~~~~~~~}
\ep
If we draw {\LARGE\tong{8}} or {\LARGE\suo{8}} first, we discard {\LARGE\wan{5}} to aim for {\jap sanshoku} of 678. 
\bp
\wan{5}\wan{6}\wan{7}\wan{8}\tong{3}\tong{3}\tong{6}\tong{7}\suo{3}\suo{4}\suo{5}\suo{6}\suo{7}~\tong{8}\\
\hfill\footnotesize{Draw~~~~~~~~~~~~~~~}
\ep
Either way, you get {\jap sanshoku} without any loss of tile efficiency. 

\newpage
\subsection{Lining pairs}\label{sec:san5}
\begin{itembox}[r]{{\jap Sanshoku} 5}
\bp
\wan{4}\wan{5}\tong{3}\tong{3}\tong{4}\tong{4}\tong{6}\tong{6}\suo{4}\suo{5}\suo{6}\suo{7}\suo{9}~\suo{8}\\
\hfill\footnotesize{Draw~~~~~~~~~~~}
\ep
\vspace{-17pt}What would you discard? \vspace{-5pt}
\end{itembox}
\noindent
As we have one block in {\jap manzu} (cracks) and two blocks in {\jap souzu} (bamboos), we need to have two blocks in {\jap pinzu} (dots). More specifically, we need the head and a group (preferably a run) in {\jap pinzu}. 
We therefore view the tiles in {\jap pinzu} not as a collection of three pairs but as a collection of one pair {\LARGE\tong{6}\tong{6}} and two side-wait protoruns {\LARGE\tong{3}\tong{4}} + {\LARGE\tong{3}\tong{4}}. 

\bigskip
From a pure tile efficiency perspective, discarding {\LARGE\tong{3}} and discarding {\LARGE\tong{4}} are equally good, and they are better than any other discards. However, there is a clear difference between the two from a perspective of hand value. 
Suppose we discard {\LARGE\tong{3}} first. If we then draw {\LARGE\tong{5}}, we will get the following hand.
\bp
\wan{4}\wan{5}\tong{3}\tong{4}\tong{4}\tong{6}\tong{6}\suo{4}\suo{5}\suo{6}\suo{7}\suo{8}\suo{9}~\tong{5}\\
\hfill\footnotesize{Draw~~~~~~~~~~~~~~~}
\ep
Discarding {\LARGE\tong{4}} makes this hand ready, but it is just a {\jap pinfu}-only hand. On the other hand, suppose we had discarded {\LARGE\tong{4}} before drawing {\LARGE\tong{5}}. 
\bp
\wan{4}\wan{5}\tong{3}\tong{3}\tong{4}\tong{6}\tong{6}\suo{4}\suo{5}\suo{6}\suo{7}\suo{8}\suo{9}~\tong{5}\\
\hfill\footnotesize{Draw~~~~~~~~~~~~~~~}
\ep
We can make this hand ready for {\jap pinfu} and {\jap sanshoku} of 456 by discarding {\LARGE\tong{6}}. 

\bigskip
Note that pre-committing to {\jap sanshoku} of 456 by discarding {\LARGE\tong{6}} before drawing {\LARGE\tong{5}} is massively inefficient. If we do that, the hand becomes a very weak hand as follows. 
\bp
\wan{4}\wan{5}\tong{3}\tong{3}\tong{4}\tong{4}\tong{6}\suo{4}\suo{5}\suo{6}\suo{7}\suo{8}\suo{9}
\ep
This hand relies too much on the possibility of drawing {\LARGE\tong{5}} first. If we draw any of {\LARGE\wan{3}\wan{6}\tong{3}\tong{4}} first, the hand will be a {\jap yaku}-less and/or bad-wait hand.

\newpage
\subsection{Golden}\label{sec:san6}
\begin{itembox}[r]{{\jap Sanshoku} 6}
\bp
\tong{2}\tong{3}\tong{4}\tong{5}\tong{7}\tong{8}\tong{9}\suo{5}\suo{5}\suo{6}\suo{7}\suo{8}\suo{9}~\wan{8}\\
\hfill\footnotesize{Draw~~~~~~~~~~~}
\ep
\vspace{-17pt}What would you discard? \vspace{-5pt}
\end{itembox}
\noindent
If we keep {\LARGE\wan{8}} and discard {\LARGE\suo{6}}, the hand becomes what is known as {\bf golden 1-away}, as follows.
\bp
\wan{8}\tong{2}\tong{3}\tong{4}\tong{5}\tong{7}\tong{8}\tong{9}\suo{5}\suo{5}\suo{7}\suo{8}\suo{9}
\ep \index{1-away (1-{\jap shanten})!golden 1-away}
It is called ``golden'' because the hand is 1-away from ready for {\jap sanshoku} \emph{and} 1-away from ready for {\jap ittsu} (Pure Straight), two of the most popular two-{\jap han yaku} in riichi mahjong. 
Drawing {\LARGE\tong{1}} or {\LARGE\tong{6}} makes the hand ready for {\jap ittsu}, whereas drawing {\LARGE\wan{7}} or {\LARGE\wan{9}} makes the hand ready for {\jap sanshoku} of 789. The following are examples of golden 1-away.
\bp
\wan{1}\wan{3}\wan{4}\wan{5}\wan{7}\wan{8}\wan{9}\tong{4}\suo{3}\suo{4}\suo{5}\fa\fa\\
\wan{5}\wan{6}\wan{7}\tong{1}\tong{2}\tong{3}\tong{4}\tong{5}\tong{6}\tong{7}\suo{3}\suo{3}\suo{6}
\ep

\newpage
\subsection{Crashing a group}\label{sec:san7}
\begin{itembox}[r]{{\jap Sanshoku} 7}
\bp
\wan{4}\wan{5}\wan{6}\tong{1}\tong{2}\tong{3}\tong{4}\tong{5}\suo{4}\suo{5}\suo{5}\suo{6}\suo{6}~\tong{3}\\
\hfill\footnotesize{Draw~~~~~~~~~~~}
\ep
\vspace{-15pt}What would you discard? \vspace{-5pt}
\end{itembox}
\noindent
Notice that the hand can be made ready for {\jap pinfu} if we discard {\LARGE\suo{5}}. However, that gives us a {\jap pinfu}-only hand. If we need an expensive hand, we could take a rather high-handed approach and crash an already complete run by discarding {\LARGE\tong{1}}. This might sound crazy, but look how good a 1-away hand it becomes.

\bp
\wan{4}\wan{5}\wan{6}\tong{2}\tong{3}\tong{3}\tong{4}\tong{5}\suo{4}\suo{5}\suo{5}\suo{6}\suo{6}
\ep
If we draw {\LARGE\tong{6}} or {\LARGE\suo{4}}, the hand becomes ready for {\jap tanyao + pinfu + sanshoku + iipeiko} (Pure Double Sequence). Drawing {\LARGE\suo{7}} also gets us {\jap tanyao + pinfu + sanshoku}, and drawing any of {\LARGE\tong{2} \tong{3} \tong{4} \tong{5}} gets us at least {\jap tanyao + pinfu}, and possibly {\jap iipeiko} as well. 

\newpage
\section{How to get {\jap ittsu}}\label{sec:itt}
	\index{ittsu@{\jap ittsu}}

{\jap Ittsu} (Pure Straight) is another popular two-{\jap han yaku}. As we will see below, even when {\jap ittsu} \emph{appears} to be a possibility, it is not always worthwhile to pursue this {\jap yaku} at the cost of tile efficiency. We will see several instances where pursuing {\jap ittsu} is and is not worth the cost.

\subsection{Two non-overlapping runs}
The key to getting {\jap ittsu} is to pay attention to {\bf two non-overlapping runs} in a given suit. For example, suppose a hand has two non-overlapping runs such as the following. 

%\bigskip 
%\begin{table}[h]\centering
%\begin{tabular}{ccc}
%{\Large\wan{1}\wan{2}\wan{3}+\wan{5}\wan{6}\wan{7}}&
%{\Large\tong{2}\tong{3}\tong{4}+\tong{6}\tong{7}\tong{8}}&
%{\Large\suo{3}\suo{4}\suo{5}+\suo{7}\suo{8}\suo{9}}\\
%\end{tabular}
%\end{table}

%{\begin{center}\Huge
%\wan{1}\wan{2}\wan{3}+\wan{5}\wan{6}\wan{7}\\ [\sep]
%\tong{2}\tong{3}\tong{4}+\tong{6}\tong{7}\tong{8}\\ [\sep]
%\suo{3}\suo{4}\suo{5}+\suo{7}\suo{8}\suo{9}\\
%\end{center}}

{\begin{center}
{\Huge \wan{1}\wan{2}\wan{3}+\wan{5}\wan{6}\wan{7}}  \\ [\sep]
{\Huge \tong{2}\tong{3}\tong{4}+\tong{6}\tong{7}\tong{8}} \\ [\sep]
{\Huge \suo{3}\suo{4}\suo{5}+\suo{7}\suo{8}\suo{9}} 
\end{center}}

\noindent
Then, as soon as we draw another non-overlapping tile in the same suit,  {\jap ittsu} is almost around the corner. 
Consider the following hand. 

\begin{itembox}[r]{{\jap Ittsu} 1}
\bp
\wan{1}\wan{2}\wan{3}\wan{5}\wan{6}\wan{7}\tong{6}\tong{8}\tong{8}\suo{2}\suo{4}\suo{6}\suo{6}~\wan{9}\\
\hfill\footnotesize{Draw~~~~~~~~~~~}
\ep
\vspace{-17pt}What would you discard? \vspace{-5pt}
\end{itembox}
\noindent
From a pure tile efficiency perspective, the best discard choice is {\LARGE\wan{9}}.
However, doing so means giving up on {\jap ittsu} and poking our way toward a bad-wait {\jap yaku}-less hand. That is not a very good path to take even if we are ahead of the game in South-4 and don't need an expensive hand.\footnote{We would still want to have at least one {\jap yaku} in a hand so that we can win it without calling riichi.}

\bigskip
We should rather treat {\LARGE\wan{9}} as a treasure; we now have a realistic possibility of getting {\jap ittsu}. Let's apply the five-block method to figure out an alternative discard.
\bmj{\huge
$ 
\underbrace{\text{\wan{1}\wan{2}\wan{3}}}
\underbrace{\text{\wan{5}\wan{6}}}
\underbrace{\text{\wan{7}\wan{9}}}
\underbrace{\text{\tong{6}\tong{8}\tong{8}}}
\underbrace{\text{\suo{2}\suo{4}\suo{6}\suo{6}}}
\nonumber
$
}\emj
We are hoping to get three blocks in {\jap manzu} (cracks) to have {\jap ittsu}, so we need one block in {\jap pinzu} (dots) and another block in {\jap souzu} (bamboos). Recall the principle that each block should have at most three tiles, which suggests we discard one tile from the block in {\jap souzu} (bamboos). The choice now boils down to discarding {\LARGE\suo{6}} or {\LARGE\suo{2}}. Recall also that the value of pairs is maximized when there are two pairs in a hand. We should thus discard {\LARGE\suo{2}}. 

\subsection{Six-tile block with intervals}
Consider different six-tile configurations where we have a chunk of six tiles with a few intervals among them. For example, consider the following six-tile blocks. 

{\begin{center}
{\Huge\wan{1}\wan{2}+\wan{4}\wan{6}+\wan{7}\wan{9}}\\ [\sep]
{\Huge\tong{1}\tong{3}+\tong{5}\tong{6}+\tong{8}\tong{9}}\\ [\sep]
{\Huge\suo{1}\suo{3}+\suo{4}\suo{6}+\suo{7}\suo{9}}
\end{center}}

\noindent
We do see {\jap ittsu} on the horizon with each of these tile chunks, but aiming for {\jap ittsu} with these blocks is not very realistic. Take the first six-tile block in {\jap manzu} (cracks), for example. Even when we draw {\LARGE\wan{8}}, we would want to discard {\LARGE\wan{1}} to have a double closed shape {\LARGE\wan{2}\wan{4}\wan{6}} and a run {\LARGE \wan{7}\wan{8}\wan{9}} rather than trying too hard to pursue {\jap ittsu}. 
With this in mind, consider the following hand. 

\begin{itembox}[r]{{\jap Ittsu} 2}
\bp
\wan{1}\wan{3}\wan{5}\wan{5}\tong{4}\tong{6}\tong{6}\suo{1}\suo{3}\suo{4}\suo{6}\suo{7}\suo{9}~\tong{7}\\
\hfill\footnotesize{Draw~~~~~~~~~~~}
\ep
\vspace{-17pt}What would you discard? \vspace{-5pt}
\end{itembox}
\noindent
Although we see a remote possibility of {\jap ittsu} in {\jap souzu} (bamboos), pursuing it requires we fill in three closed-wait protoruns in {\jap souzu}: {\LARGE\suo{1}\suo{3}}, {\LARGE\suo{4}\suo{6}}, and {\LARGE\suo{7}\suo{9}}.

\bigskip
It would be more practical to discard {\LARGE\suo{1}} and then {\LARGE\suo{9}}; we would consider the tile blocks in {\jap souzu} as a collection of two side-wait protoruns: {\LARGE\suo{3}\suo{4}} and {\LARGE\suo{6}\suo{7}} + two redundant terminal tiles: {\LARGE\suo{1} \suo{9}}, rather than considering it as a collection of three closed-wait protoruns.

\bigskip
\subsection{Run + side-wait protorun}

At one step prior to getting two non-overlapping runs, we may have one run and a non-overlapping side-wait protorun in a given suit. The following blocks are examples of such run + side-wait protorun combinations.

\bigskip
{\begin{center}
{\Huge \wan{3}\wan{4}\wan{5}+\wan{7}\wan{8}}  \\ [\sep]
{\Huge \tong{3}\tong{4}+\tong{6}\tong{7}\tong{8}} \\ [\sep]
{\Huge \suo{4}\suo{5}+\suo{7}\suo{8}\suo{9}} 
\end{center}}

\noindent
When we have a combination like these, a draw of {\LARGE\wan{1}} or {\LARGE\wan{2}} (left example), {\LARGE\tong{1}} or {\LARGE\tong{9}} (middle example), {\LARGE\suo{1}} or {\LARGE\suo{2}} (right example) generates a realistic probability of getting {\jap ittsu}. Below are the resulting tile blocks in each instance. You can pursue {\jap ittsu} with any of these blocks. 

\bigskip 
\begin{table}[h]\centering
\begin{tabular}{ccc}
\hspace{-15pt}{\LARGE\wan{1}+\wan{3}\wan{4}\wan{5}+\wan{7}\wan{8}}&
{\LARGE\tong{1}+\tong{3}\tong{4}+\tong{6}\tong{7}\tong{8}}&
{\LARGE\suo{1}+\suo{4}\suo{5}+\suo{7}\suo{8}\suo{9}}\\ [\sep]
\hspace{-15pt}{\LARGE\wan{2}+\wan{3}\wan{4}\wan{5}+\wan{7}\wan{8}}&
{\LARGE\tong{3}\tong{4}+\tong{6}\tong{7}\tong{8}+\tong{9}}&
{\LARGE\suo{2}+\suo{4}\suo{5}+\suo{7}\suo{8}\suo{9}}\\
\end{tabular}
\end{table}

\bigskip
However, if the protorun is instead a closed-wait or an edge-wait one, the chance of getting {\jap ittsu} is much diminished. The following blocks are examples of such run + closed- or edge-wait protorun combinations. You may not want to pursue {\jap ittsu} with these blocks. 

%\bigskip
%{\LARGE \wan{3}\wan{4}\wan{5}+\wan{8}\wan{9}}  \hfill 
%{\LARGE \tong{1}\tong{2}+\tong{6}\tong{7}\tong{8}} \hfill
%{\LARGE \suo{1}\suo{3}+\suo{7}\suo{8}\suo{9}} 

\bigskip
{\begin{center}
{\Huge \wan{3}\wan{4}\wan{5}+\wan{8}\wan{9}}  \\ [\sep]
{\Huge \tong{1}\tong{2}+\tong{6}\tong{7}\tong{8}} \\ [\sep]
{\Huge \suo{1}\suo{3}+\suo{7}\suo{8}\suo{9}} 
\end{center}}

\bigskip
\noindent If the run becomes a bulging float block, you may want to give up on {\jap ittsu} and discard the closed- or edge-wait protorun part. With this in mind, consider the following 2-away hand.

\bigskip
\begin{itembox}[r]{{\jap Ittsu} 3}
\bp
\wan{1}\wan{3}\wan{4}\wan{5}\wan{8}\wan{9}\tong{5}\tong{5}\suo{3}\suo{3}\suo{6}\bai\bai~\wan{4}\\
\hfill\footnotesize{Draw~~~~~~~~~~~}
\ep
\vspace{-17pt}What would you discard? \vspace{-5pt}
\end{itembox}
\noindent
Now that we drew a tile that creates a bulging float block in {\jap manzu} (cracks), it is about time to give up on {\jap ittsu}. Discarding {\LARGE\wan{1}} allows the hand to accept 9 kinds--25 tiles; if we stick with {\jap ittsu} and discard {\LARGE\wan{4}}, the hand can accept only 4 kinds--10 tiles. 

\bigskip
Moreover, although the hand is 2-away from ready, it is 3-away from {\jap ittsu}. Pursuing {\jap ittsu} with a hand like this is not very practical.

\bigskip
\subsection{{\jap Ittsu} vs. side wait}
As we saw with {\jap sanshoku} hands, we often face a choice between pursuing {\jap yaku} and keeping a side-wait protorun. Consider the following 1-away hand. 

\begin{itembox}[r]{{\jap Ittsu} 4}
\bp
\wan{3}\wan{4}\wan{5}\wan{5}\wan{6}\wan{7}\wan{8}\wan{9}\tong{4}\tong{5}\tong{5}\suo{1}\suo{1}~\wan{1}~\tong{5}\\
\hfill\footnotesize{Draw~~{\jap Dora}~~~~~~~}
\ep \index{1-away (1-{\jap shanten})!perfect 1-away}
\vspace{-15pt}What would you discard? \vspace{-5pt}
\end{itembox}
\noindent
If we discard {\LARGE\wan{1}} we drew, the hand is a perfect 1-away hand; the final wait can always be side wait. On the other hand, if we discard {\LARGE\wan{5}}, we have a confirmed {\jap ittsu} hand. Which option should we choose?

\bigskip
If we compare tile acceptance counts for the two scenarios, the option of confirming side wait is slightly better (6 kinds--18 tiles vs. 5 kinds--16 tiles). However, doing so means we give up on {\jap ittsu}. Moreover, giving up on {\jap ittsu} means that we can never call {\jap pon} or {\jap chii} with this hand because there is no {\jap yaku} in the hand. On the other hand, the second option allows us to call {\jap pon} on {\LARGE\tong{5}\suo{1}} or call {\jap chii} on {\LARGE\wan{2}}. 
Even though the kinds and the number of acceptable tiles are smaller, the second option would be more efficient if we take melding into account. 

\newpage
\section{How to get {\jap pinfu}} \index{pinfu@{\jap pinfu}}\label{sec:pinfu}

%%%%%% Stopped here (6 April, 2017)

Although {\jap pinfu} is only worth one {\jap han}, the requirements to claim {\jap pinfu} are rather demanding. The key to getting {\jap pinfu} is to build side-wait protoruns even at the cost of tile efficiency. Consider the following hand. 
\begin{itembox}[r]{{\jap Pinfu} 1}
\bp
\wan{2}\wan{3}\wan{5}\wan{5}\wan{8}\wan{9}\tong{4}\tong{7}\tong{8}\suo{2}\suo{3}\suo{4}\suo{6}~\wan{1}\\
\hfill\footnotesize{Draw~~~~~~~~~~~}
\ep
\vspace{-17pt}What would you discard? \vspace{-5pt}
\end{itembox}
\noindent
We already have five tile blocks in this hand. From a pure tile efficiency perspective, discarding one of the two floating tiles {\LARGE\tong{4}} or {\LARGE\suo{6}} is the best. However, doing so significantly reduces our chance of getting {\jap pinfu}. If we aim for {\jap pinfu} we should discard the edge-wait protorun {\LARGE\wan{8}\wan{9}} and keep the two floating tiles, which we hope may grow into a side-wait protorun. 

\bigskip
Suppose we discarded {\LARGE\wan{8}}, then we drew {\LARGE\tong{5}}, after which we discarded {\LARGE\wan{9}}. Now the hand is 1-away again, this time with two side-wait protoruns. Suppose further that we drew {\LARGE\xi}. 
\begin{itembox}[r]{{\jap Pinfu} 2}
\bp
\wan{1}\wan{2}\wan{3}\wan{5}\wan{5}\tong{4}\tong{5}\tong{7}\tong{8}\suo{2}\suo{3}\suo{4}\suo{6}~\xi\\
\hfill\footnotesize{Draw~~~~~~~~~~~}
\ep
\vspace{-17pt}What would you discard? \vspace{-5pt}
\end{itembox}
\noindent
We should keep {\LARGE\xi} as a safe tile and discard {\LARGE\suo{6}}. It is true that keeping {\LARGE\suo{6}} has an advantage; if we draw {\LARGE\suo{5}}, we will get a 3-way side-wait block in {\jap souzu} (bamboos). Even a draw of {\LARGE\suo{7}} improves this hand slightly. This is because having {\LARGE\tong{4}\tong{5}} and {\LARGE\tong{7}\tong{8}} is not very efficient due to the overlap of the waiting tiles; both blocks wait for {\LARGE\tong{6}}. It would be better to have {\LARGE\suo{6}\suo{7}} and {\LARGE\tong{4}\tong{5}}, rather than having {\LARGE\tong{4}\tong{5}} and {\LARGE\tong{7}\tong{8}}.

\bigskip
However, keeping {\LARGE\suo{6}} comes at a cost. Even when we draw {\LARGE\suo{5}}, we will then have to discard {\LARGE\tong{4}\tong{5}} or {\LARGE\tong{7}\tong{8}}, possibly against an opponent's riichi. 
Therefore, once we get a 1-away with two side-wait protoruns ({\bf side \& side 1-away}; {\jap ryanmen-ryanmen 1-shanten}), we should try to keep a safety tile in the hand.
\index{1-away (1-{\jap shanten})!side \& side 1-away}

\bigskip
Even when we draw a tile that makes a hand perfect 1-away, we may still want to have a safety tile. For example, drawing any of {\LARGE\tong{4}\tong{5}\tong{7}\tong{8}} makes the hand above perfect 1-away. 
\index{1-away (1-{\jap shanten})!perfect 1-away}
Although perfect 1-away is better than side \& side 1-away in terms of tile acceptance, a perfect 1-away hand can end up not having {\jap pinfu} because a set can emerge in the hand. 

\vfill
There is one exception to this, however. If the floating tile leaves a possibility of enhancing the hand value by at least three {\jap han}, it is OK to keep it instead of a safety tile. Consider the following hand. 
\begin{itembox}[r]{{\jap Pinfu} 3}
\bp
\wan{1}\wan{2}\wan{3}\wan{5}\wan{7}\wan{8}\wan{9}\tong{3}\tong{4}\tong{7}\tong{7}\suo{2}\suo{3}~\xi~\wan{4}\\
\hfill\footnotesize{Draw~~{\jap Dora}~~~~~~~}
\ep
\vspace{-17pt}What would you discard? \vspace{-5pt}
\end{itembox}

\bigskip
\noindent Keeping {\LARGE\xi} is safer, but keeping {\LARGE\wan{5}} leaves the possibilities of getting {\jap ittsu} and having {\jap dora}, possibly at the same time. In this case, we would rather discard {\LARGE\xi}. 

\newpage

\subsubsection{Building the head}
To claim {\jap pinfu}, the head must be a pair of number tiles or valueless wind tiles. Therefore, try not to discard terminals or valueless wind tiles lightly when having a {\jap pinfu} hand. Keep this in mind especially when a hand is lacking any pair. Assuming you are the South player in the 1st turn in East-1, consider the following hand. 
\begin{itembox}[r]{{\jap Pinfu} 4}
\bp
\wan{1}\wan{4}\wan{6}\tong{2}\tong{3}\tong{6}\tong{7}\tong{9}\suo{5}\suo{6}\suo{6}\suo{7}\xi\fa
\ep
\vspace{-10pt}What would you discard? \vspace{-5pt}
\end{itembox}

\bigskip
\noindent The hand has a potential to have {\jap pinfu}, so we should not discard any of {\LARGE\wan{1}\tong{9}\xi} at this point. All of these three tiles may appear useless, but they can be the head of a {\jap pinfu} hand when any of them grows into a pair. On the other hand, the value tile {\LARGE\fa} cannot be the head of {\jap pinfu}. We should discard {\LARGE\fa} in this case. 

\newpage
\section{How to get {\jap honitsu}}\label{sec:hon}
	\index{honitsu@{\jap honitsu} (Half Flush)}
Going for {\jap honitsu} (Half Flush) can be a good way to achieve high hand values. As we can combine {\jap honitsu} with many other {\jap yaku}, including {\jap fanpai}, {\jap toitoi}, {\jap chanta} (Outside Hand), {\jap ittsu}, among others,\footnote{Technically speaking, {\jap honitsu} can be combined with {\jap chiitoitsu}, {\jap shousangen} (Little Three Dragons), {\jap honroutou} (All Terminals and Honors), {\jap pinfu}, {\jap iipeiko}, {\jap ryanpeiko}, {\jap san ankou} (Three Concealed Triplets), and {\jap san kantsu} (Three Quads) as well.}
we can aim for {\jap mangan} relatively easily.
The fact that {\jap honitsu} is worth two {\jap han} even when we open our hand means we can also enhance the speed by melding without making our hand too cheap.

\subsection{Conditions to go for {\jap honitsu}}
When judging whether to go for {\jap honitsu} or not, we should consider two factors --- five-block potential and hand values with and without {\jap honitsu}.

\subsubsection{1. Five-block potential}
\noindent The most important factor to consider is whether or not your hand has  five tile blocks or block candidates (i.e., floating tiles) necessary for {\jap honitsu}. Assuming you are the South player in the 6th turn in East-1, consider the following hand. 

\begin{itembox}[r]{{\jap Honitsu} 1}
\bp
\wan{1}\wan{5}\wan{5}\wan{7}\wan{9}\tong{8}\suo{2}\suo{4}\dong\dong\xi\zhong\zhong
\ep
\vspace{-10pt}Would you go for {\jap honitsu}? \vspace{-5pt}
\end{itembox}
\noindent
In order to figure out if it is practical to pursue {\jap honitsu} with this hand, let's apply the five-block method. 
\bmj{\huge
$ 
\underbrace{\text{\wan{5}\wan{5}}}
\underbrace{\text{\wan{7}\wan{9}}}
\underbrace{\text{\dong\dong}}
\underbrace{\text{\zhong\zhong}}
\underbrace{\text{\wan{1}\xi}}
\text{\tong{8}\suo{2}\suo{4}}
\nonumber
$
}\emj
We can count on the two pairs of {\jap fanpai}, {\LARGE\dong\dong} and {\LARGE\zhong\zhong}, to be two tile blocks, a pair of {\LARGE\wan{5}\wan{5}} and a protorun {\LARGE\wan{7}\wan{9}} to be another two blocks, yielding four blocks in total. In addition, we can reasonably expect either of the two floating tiles {\LARGE\xi}{\LARGE\wan{1}} to produce the fifth block. Therefore, you can go for {\jap honitsu} with this hand. 

\bigskip
In addition, the tiles that are made redundant in the hand if we choose {\jap honitsu} are an isolated {\LARGE\tong{8}} and a closed-wait protorun {\LARGE\suo{2}\suo{4}}. Keeping these tiles would not make this hand particularly more efficient anyway, so we can go for {\jap honitsu} without much hesitation. 

\bigskip
Even when we have a side-wait protorun or a pair to discard, we may still want to go for {\jap honitsu}. For example, with the following two hands, you should go for {\jap honitsu} even though doing so means you have to discard a side-wait protorun or a pair. 
\bp
\wan{2}\wan{3}\tong{1}\tong{3}\tong{4}\tong{5}\tong{5}\tong{6}\tong{8}\tong{8}\xi\bai\bai\bai\\
\tong{4}\tong{4}\suo{1}\suo{2}\suo{4}\suo{4}\suo{5}\suo{5}\suo{9}\dong\nan\nan\bei\zhong
\ep

\subsubsection{2. Hand value}

Another factor to consider is hand value comparison with and without {\jap honitsu}. If your hand does not have any {\jap yaku} potential (e.g., pair or set of {\jap fanpai}) other than {\jap honitsu}, you may end up getting a {\jap honitsu}-only hand, which is very cheap (2000 or 2600 points). In such situations, you should not aim for {\jap honitsu}; you should try to make the hand ready without melding and go for riichi. Assuming you are the South player in East-1, consider the following hand.
\begin{itembox}[r]{{\jap Honitsu} 2}
\bp
\wan{1}\wan{2}\wan{4}\wan{6}\wan{8}\wan{8}\tong{4}\tong{5}\suo{7}\suo{7}\xi\xi\bei\bei
\ep
\vspace{-10pt}Would you go for {\jap honitsu}? \vspace{-5pt}
\end{itembox}
\noindent
Since West and North are both valueless wind tiles, this hand is likely to become {\jap honitsu}-only if you decide to go for {\jap honitsu}. Although this hand has five tile blocks necessary for {\jap honitsu}, you should not go for {\jap honitsu}. 

\bigskip
At the same time, when the hand value is sufficiently high ($\geq$ 5200) \emph{without} {\jap honitsu}, you should not go for {\jap honitsu} at the cost of tile efficiency. Assuming you are the South player in East-1, consider the following hand.
\begin{itembox}[r]{{\jap Honitsu} 3}
\bp \vspace{-10pt}
\hspace{-145pt}{\footnotesize\color{red!75!black} Red}\\ \vspace{-16pt}
\wan{1}\wan{1}\wan{4}\rfw\wan{7}\tong{3}\tong{4}\fa\fa\fa\xi~~\nan\nan\rnan
\ep
\vspace{-10pt}What would you discard? \vspace{-5pt}
\end{itembox}

\bigskip
\noindent This hand is worth 5200 points without {\jap honitsu} (Seat Wind + Green Dragon + red five), so you should discard {\LARGE\wan{7}} to maintain a side \& side 1-away status. If {\LARGE\nan} were not your seat wind, you should go for {\jap honitsu}.

\newpage
\subsection{Discard}
When pursuing {\jap honitsu}, pay attention to the order of your discards. 
Consider the following hand. You called {\jap pon} on {\LARGE\bai} just now, deciding what to discard. 
\begin{itembox}[r]{{\jap Honitsu} 4}
\bp
\wan{1}\wan{1}\wan{2}\wan{6}\wan{8}\tong{7}\suo{3}\suo{5}\bei\bei\zhong~\bai\bai\rbai
\ep
\vspace{-10pt}What would you discard? \vspace{-5pt}
\end{itembox}

\bigskip
\noindent You should pursue {\jap honitsu}, so {\LARGE\tong{7}\suo{3}\suo{5}} are your discard candidates. You will discard all three of them eventually, but you should discard them in a way that looks less obvious that you are collecting tiles in {\jap manzu} (cracks). 
If you discard {\LARGE\tong{7}} first then {\LARGE\suo{5}} next, the opponents might (correctly) guess that you are doing {\jap honitsu} with {\jap manzu} tiles. 
In particular, the Left player may stop discarding tiles in {\jap manzu} that you could call {\jap chii} on. You should thus discard {\LARGE\suo{5}} first then {\LARGE\suo{3}} next, so that the opponents cannot know if you are collecting {\jap manzu} or {\jap pinzu} (dots). They will eventually find out that you are collecting {\jap manzu}, but you should delay that as much as possible. 

\newpage
\subsection{Melding}
When you start melding with a {\jap honitsu} hand, try to leave the possibility of achieving the maximum hand value. Assuming you are the South player in the 6th turn in East-1, consider the following hand.

\begin{itembox}[r]{{\jap Honitsu} 5}
\bp
\wan{9}\wan{9}\tong{2}\tong{3}\tong{5}\tong{7}\tong{8}\tong{9}\xi\bei\bei\fa\fa
\ep
\vspace{-10pt}Which tile would you call? \vspace{-5pt}
\end{itembox}

\bigskip
\noindent With this hand, do not start melding with a {\jap chii} of {\LARGE\tong{4}} or a {\jap pon} of {\LARGE\bei} if you are playing without red fives; you may end up with a very cheap (2000 points) {\jap honitsu}-only hand. 
Suppose you managed to call {\jap pon} on {\LARGE\fa}, resulting in the following hand. 

\bp
\wan{9}\tong{2}\tong{3}\tong{5}\tong{7}\tong{8}\tong{9}\xi\bei\bei~\fa\rfa\fa
\ep

\noindent Calling {\jap chii} on {\LARGE\tong{4}} or {\jap pon} on {\LARGE\bei} is still not ideal. The only melding you should do is to call {\jap chii} on {\LARGE\tong{1}} to have the following hand.

\bp
\tong{5}\tong{7}\tong{8}\tong{9}\xi\bei\bei~\rtong{1}\tong{2}\tong{3}~\fa\rfa\fa
\ep

Notice that the two floating tiles {\LARGE\tong{5} \xi} allow us to envision two possibilities of getting a 7700 hand. 
On the one hand, if you draw {\LARGE\xi} or call {\jap pon} on {\LARGE\bei}, you get {\jap honitsu} + Green Dragon + {\jap chanta} (Outside Hand), as follows.

\bp
\tong{7}\tong{8}\tong{9}\xi\xi\bei\bei~\rtong{1}\tong{2}\tong{3}~\fa\rfa\fa\\
\tong{7}\tong{8}\tong{9}\xi~\bei\bei\rbei~\rtong{1}\tong{2}\tong{3}~\fa\rfa\fa
\ep

On the other hand, if you draw {\LARGE\tong{4}} or {\LARGE\tong{6}}, the hand will be ready for {\jap honitsu} + Green Dragon + {\jap ittsu}. 

\bp
\tong{4}\tong{5}\tong{7}\tong{8}\tong{9}\bei\bei~\rtong{1}\tong{2}\tong{3}~\fa\rfa\fa\\
\tong{5}\tong{6}\tong{7}\tong{8}\tong{9}\bei\bei~\rtong{1}\tong{2}\tong{3}~\fa\rfa\fa
\ep

\newpage
\section{How to get {\jap toitoi} / {\jap chiitoitsu}} \label{sec:toi}

\subsection{{\jap Toitoi} vs. {\jap chiitoitsu}}
	\index{chiitoitsu@{\jap chiitoitsu} (Seven Pairs)}
	\index{toitoi@{\jap toitoi} (All Triplets)}
When pursuing {\jap chiitoitsu} (Seven Pairs), you may find yourself standing at a crossroad between {\jap chiitoitsu} and {\jap toitoi} (All Triplets). Specifically, what should we do when one of the pairs in a 1-away {\jap chiitoitsu} hand becomes a set? Assuming you are the South player in the 6th turn in East-1, consider the following hand. 

\bigskip
\begin{itembox}[r]{{\jap Toitoi} vs. {\jap chiitoitsu}}
\bp
\wan{7}\wan{7}\wan{8}\wan{8}\tong{1}\tong{3}\tong{3}\suo{5}\suo{5}\suo{7}\fa\fa\bei~\tong{3}\\
\hfill\footnotesize{Draw~~~~~~~~~~~}
\ep
\vspace{-17pt}What would you discard? \vspace{-5pt}
\end{itembox}

\bigskip
\noindent If we discard {\LARGE\tong{3}} that we drew, the hand is 1-away from ready for {\jap chiitoitsu}, accepting {\LARGE\tong{1}\suo{7}\bei} (3 kinds--9 tiles). On the other hand, if we keep it and discard {\LARGE\suo{7}} instead, the hand is still a 1-away {\jap chiitoitsu} hand, albeit with smaller tile acceptance. However, doing so makes the hand also 2-away from ready for {\jap toitoi} and possibly {\jap su anko} (Four Concealed Triplets). 

\bigskip
\noindent Judgement criteria for a choice of this kind are summarized as follows.
\bigskip
\begin{itembox}[c]{{\jap Toitoi} vs. {\jap chiitoitsu}}
Choose {\jap chiitoitsu} in the following situations.
\be\itemsep.1pt
\i There is a futile pair in your hand.
\i There is no pair of value tiles in your hand.
\i There are three or more pairs of simple tiles between 3 and 7 in your hand.
\ee
\end{itembox}

\bigskip
\noindent The first condition is by far the most important one. With the current hand example, if the opponents have already discarded two tiles of {\LARGE\wan{7}}, the pair of {\LARGE\wan{7}} in the hand is a {\bf futile pair} (dead pair) that will never become a set. If there is one or more futile pair in your hand, you must stick with {\jap chiitoitsu}. If not, you can go for {\jap toitoi}. 

\bigskip
In addition, you may also want to take into account the second and the third conditions. Specifically, without having a pair of value tiles ({\jap fanpai}), you may end up with a {\jap toitoi}-only hand (2600 or even 2000 points). With one pair of value tiles, you can aim for 5200 with {\jap toitoi}; with two pairs of them, you can aim for {\jap mangan}. 

\bigskip
The third factor to consider is whether there are \emph{not} three or more pairs of simple tiles between 3 and 7. Consider the following hand. 
\bp
\wan{6}\wan{7}\wan{7}\tong{5}\tong{5}\suo{2}\suo{2}\suo{6}\suo{6}\bai\bai\fa\zhong
\ep
Suppose you start melding by calling {\jap pon} on {\LARGE\bai},\footnote{Calling {\jap pon} on {\bai} makes this 1-away {\jap chiitoitsu} hand 2-away from ready for {\jap toitoi}. Doing so would be acceptable if the remaining pairs were not simple tiles between 3 and 7.} then get another {\jap pon} on {\LARGE\suo{2}}, resulting in the following hand.
\bp
\wan{7}\wan{7}\tong{5}\tong{5}\suo{6}\suo{6}\zhong~\suo{2}\suo{2}\rsuo{2}~\bai\rbai\bai
\ep
Since the remaining three pairs are all simple tiles between 3 and 7, the hand advancement often stops here. Because of their high versatility, simple tiles between 3 and 7 are very likely to be used by the opponents. 

\newpage
\subsection{Standard hand vs. {\jap chiitoitsu}}
Another kind of crossroad is between {\jap chiitoitsu} and standard hand. Assuming you are the South player in the 6th turn in East-1, consider the following hand. 
\bigskip
\begin{itembox}[r]{{\jap Pinfu} vs. {\jap chiitoitsu}}
\bp
\wan{4}\wan{5}\wan{6}\wan{6}\tong{4}\tong{5}\tong{8}\tong{8}\suo{6}\suo{6}\suo{7}\bei\bei~\wan{5}\\
\hfill\footnotesize{Draw~~~~~~~~~~~}
\ep
\vspace{-17pt}What would you discard? \vspace{-5pt}
\end{itembox}

\bigskip
\noindent As we draw {\LARGE\wan{5}}, we now have five pairs in the hand, making it 1-away from ready for {\jap chiitoitsu}. However, the hand is also 2-away from ready if we interpret this hand as a standard hand. 

\bigskip
When a hand has this many side-wait protoruns, it makes more sense to view it as a standard hand rather than as a {\jap chiitoitsu} hand. To figure out what tile to discard, let's apply the five-block method. 

\bmj{\Huge
$ 
\underbrace{\text{\wan{4}\wan{5}\wan{5}\wan{6}\wan{6}}}_{\text{\small 2}}
\underbrace{\text{\tong{4}\tong{5}}}
\underbrace{\text{\tong{8}\tong{8}}}
\underbrace{\text{\suo{6}\suo{6}\suo{7}}}
\text{\bei\bei}
\nonumber
$
}\emj
Since we already have five tile blocks in simple tiles, the pair of {\LARGE\bei} is redundant. Discarding {\LARGE\bei} means we are giving up on {\jap chiitoitsu}, but we are maximizing tile acceptance to make the hand ready for {\jap tanyao} as soon as possible. Doing so leaves a decent chance of getting {\jap pinfu} and {\jap iipeiko} as well. The expected hand value will actually be higher if we give up on {\jap chiitoitsu}.

\bigskip
On the other hand, when a hand has few side-wait protoruns but has several pairs, you should pursue a pair-based (set-based) hand rather than a run-based hand. Assuming you are the South player in the 6th turn in East-1, consider the following hand. 

\bigskip
\begin{itembox}[r]{Run-based hand vs. set-based hand}
\bp
\wan{1}\wan{1}\wan{2}\wan{2}\tong{4}\tong{5}\suo{2}\suo{4}\suo{9}\suo{9}\dong\nan\bai~\bai\\
\hfill\footnotesize{Draw~~~~~~~~~~~}
\ep
\vspace{-17pt}What would you discard? \vspace{-5pt}
\end{itembox}

\bigskip
\noindent Although the hand has one side-wait protorun, we would not be very happy if it were to evolve into a complete run; we may end up with a very cheap hand with a bad wait. Alternatively, you should pursue {\jap chiitoitsu} or {\jap toitoi} with a hand like this. Let's apply the five-block method to figure out what tile to discard.

\bmj{\Huge
$ 
\underbrace{\text{\wan{1}\wan{1}}}
\underbrace{\text{\wan{2}\wan{2}}}
\underbrace{\text{\suo{9}\suo{9}}}
\underbrace{\text{\bai\bai}}
\text{\tong{4}\tong{5}\suo{2}\suo{4}\dong\nan}
\nonumber
$
}\emj
Since we intend to build four blocks using the four pairs in the hand, we only need one more block from the rest of the tiles: {\LARGE\tong{4}\tong{5}\suo{2}\suo{4}\dong\nan}. Of these six tiles, {\LARGE\dong\nan\suo{2}} are clearly more valuable than others because of their \emph{low} versatilities. On the other hand, {\LARGE\tong{4}\tong{5}\suo{4}} are less useful for you because these tiles have high versatility for the opponents. We should discard {\LARGE\tong{4}} or {\LARGE\suo{4}} first, keeping {\LARGE\tong{5}} just in case we draw a red {\LARGE\rfd} (in which case the hand is 1-away from ready for {\jap chiitoitsu}).




